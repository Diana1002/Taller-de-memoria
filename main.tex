\documentclass{article}
\usepackage[utf8]{inputenc}
\usepackage[spanish]{babel}
\usepackage{listings}
\usepackage{graphicx}
\graphicspath{ {images/} }
\usepackage{cite}

\begin{document}

\begin{titlepage}
    \begin{center}
        \vspace*{1cm}
            
        \Huge
        \textbf{NOCIONES DE LA MEMORIA DEL COMPUTADOR}
            
        \vspace{0.5cm}
        \LARGE
        Taller de Memoria
            
        \vspace{1.5cm}
            
        \textbf{DIANA LUCIA BAEZA RUIZ}
            
        \vfill
            
        \vspace{0.8cm}
            
        \Large
        Despartamento de Ingeniería Electrónica y Telecomunicaciones\\
        Universidad de Antioquia\\
        Medellín\\
        Septiembre de 2020
            
    \end{center}
\end{titlepage}

\tableofcontents
\newpage

\section{¿Qué es la memoria del computador?} \label{contenido}

La palabra "Memoria" la usamos para designar las partes del computador que sirve como soporte para el manejo todos los datos y programas que se utilizan mientras está operando. La memoria es fundamental para que funcione correctamente el computador, entre otras cosas permite que pueda arrancar.
Es la zona de trabajo donde se guardan temporalmente las órdenes que debe ejecutar y los datos que utilizan dichas órdenes en el computaodr.\cite{1website}

\section{Mencione los tipos de memoria que conoce y haga una descripción de cada tipo}

Podemos clasificar a la memoria en dos tipos basicos; La memoria R.A.M(Random Access Memory) y la memoria R.O.M(Read Only Memory).

\subsection{R.A.M:} \label{contenido} Lo que en Español significa memoria de acceso aleatorio. Es el área donde el microprocesador realiza las diferentes operaciones. Esta memoria almacena los programas que ejecuta el usuario y los datos con los que trabaja los programas.
Es un tipo de memoria volatil, todo lo contrario al disco duro, porque almacena temporalmente los datos y cuando el computador se apaga todo lo almacenado se pierde. 
Cuando se abre un archivo o programa estos de cargan en la R.A.M para poder ser utilizado por el microprocesador. Incluso el sistema operativo cuando se carga lo hace en la R.A.M.

\vspace{1cm}

\textbf{La memoria R.A.M. se clasifica en tres tipos:}
\begin{itemize}
    \item Memoria caché: Es un tipo de memoria volatil, pero rapida. Su papel es almacenar instrucciones y datos a los que el procesador debe acceder continuamente. 
    Esta diseñado para almacenar temporalmen aquellos datos a los que el sistema necesita tener un acceso instantaneo para el procesador, pues se trata de información relevante que debe estar a la mano.
    La memoria caché es 5 o 6 veces mas rapida que la R.A.M. por lo tanto más costosa, pero con menos capacidad que la R.A.M.\cite{2website}
     \item Ttearjeta de memoria R.A.M.
    \item Memoria R.A.M de videos.
\end{itemize}

\subsection{R.O.M:} \label{contenido} (Read Only Memory) que en Español significa Memoria de solo lectura y como su nombre lo indica es una memoria que solo puede leerse sin modificar su contenido. Es un tipo de memoria no volatil. Se utiliza para ayudar a realizar las diferentes operaciones de arranque del sistema.
Es mucho mas lenta y con menos capacidad que la R.A.M.

\section{Describa la manera como se gestiona la memoria en un computador}

Gestionar la memoria consiste en llevar un registro de las partes de la memoria que se estan utilizando y las que no, para asignarle espacio en la memoria a los procesos que se estan ejecutando y liberarlo cuando terminen.
La gestión de la memoria es impotante para optimizar el espacio y poder intercambiar con los programas que van hacer ejecutados del disco duro a la memoria. Tembién facilita espacio para caada proceso y controla que no se trabaje en zonas de la memoria que no han sido asignadas.

\vspace{0.5cm}

Cuando encendemos el computador, el circuito de control llamado Bios inicia una revision en el sistema para comprobar la existencia y buen funcionamiento de la R.A.M. y otras secciones de la maquina. Luego, busca en la unidad de almacenamiento permanente o Disco duro el sistema operativo para que sea cargado en la R.A.M. Entonces, el computador queda funcionando bajo las instrucciones de este sistema.
Los programas se comunican con el microprocesador y permiten realizar procesos según la necesidad del usuario. El trabajo de movimiento, comparación y depuración de datos es efectua en la memoria R.A.M. \cite{3website}

\section{¿Qué hace que una memoria sea más rápida que otra? ¿Por qué esto es importante?} \label{contenido}

El desempeño de memoria consiste en la relación entre velocidad y latencia(Ambas estan extremadamente relacionadas). Cuanto más rápido es la velocidad de la memoria, más rápido funciona el computador.
La capacidad juega un papel muy importante, ya que a mayor capacidad, mejor rendimiento
La velocidad de la memoria se mide por el intervalo de tiempo que transcurre desde que la maquina recibe a una orden hasta que la ejecuta,este intervalo se conoce como latencia.

La frecuencia es otro de los elementos más importantes de la memoria RAM, ya que cuanto mayor sea la velocidad de la memoria, más rápido podrá trabajar los datos.

\newpage




\bibliographystyle{IEEEtran}
\bibliography{references}
\citep{Programación}
\end{document}
