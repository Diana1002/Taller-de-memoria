\documentclass{article}
\usepackage[utf8]{inputenc}
\usepackage[spanish]{babel}
\usepackage{listings}
\usepackage{graphicx}
\graphicspath{ {images/} }
\usepackage{cite}

\begin{document}

\begin{titlepage}
    \begin{center}
        \vspace*{1cm}
            
        \Huge
        \textbf{NOCIONES DE LA MEMORIA DEL COMPUTADOR}
            
        \vspace{0.5cm}
        \LARGE
        Taller de Memoria
            
        \vspace{1.5cm}
            
        \textbf{DIANA LUCIA BAEZA RUIZ}
            
        \vfill
            
        \vspace{0.8cm}
            
        \Large
        Despartamento de Ingeniería Electrónica y Telecomunicaciones\\
        Universidad de Antioquia\\
        Medellín\\
        Septiembre de 2020
            
    \end{center}
\end{titlepage}

\tableofcontents

\section{Sección introductoria}
Esta es la primera sección, podemos agregar algunos elementos adicionales y todo será escrito correctamente. Más aún, si una palabra es demasiado larga y tiene que ser truncada, babel tratará de truncarla correctamente dependiendo del idioma.

\section{¿Qué es la memoria del computador?} \label{contenido}

La palabra "Memoria" la usamos para designar las partes del computador que sirve como soporte para el manejo todos los datos y programas que se utilizan mientras está operando. La memoria es fundamental para que funcione correctamente el computador, entre otras cosas permite que pueda arrancar.
Es la zona de trabajo donde se guardan temporalmente las órdenes que debe ejecutar y los datos que utilizan dichas órdenes en el computaodr.\cite{dirac}

\section{Mencione los tipos de memoria que conoce y haga una descripción de cada tipo}

Podemos clasificar a la memoria en dos tipos basicos; La memoria R.A.M(Random Access Memory) y la memoria R.O.M(Read Only Memory)

\subsection{R.A.M:} \label{contenido} Lo que en Español significa memoria de acceso aleatorio. Es el área donde el microprocesador realiza las diferentes operaciones. Esta memoria almacena los programas que ejecuta el usuario y los datos con los que trabaja los programas.
Es un tipo de memoria volatil, todo lo contrario al disco duro, porque almacena temporalmente los datos y cuando el computador se apaga todo lo almacenado se pierde. 
Cuando se abre un archivo o programa estos de cargan en la R.A.M para poder ser utilizado por el microprocesador. Incluso el sistema operativo cuando se carga lo hace en la R.A.M.

\textbf{La memoria R.A.M. se clasifica en tres tipos:}
\begin{itemize}
    \item Memoria caché
    \item Tarjeta de memoria R.A.M.
    \item Memoria R.A.M de videos
\end{itemize}

\begin{lstlisting}
#include <stdio.h>
#define N 10
/* Block
 * comment */

int main()
{
    int i;

    // Line comment.
    puts("Hello world!");
    
    for (i = 0; i < N; i++)
    {
        puts("LaTeX is also great for programmers!");
    }

    return 0;
}
\end{lstlisting}

A continuación se presenta el logo de C++ Figura (\ref{fig:cpplogo})

\begin{figure}[h]
\includegraphics[width=4cm]{cpplogo.png}
\centering
\caption{Logo de C++}
\label{fig:cpplogo}
\end{figure}

En la sección de teoremas (\ref{contenido})

\section{Conclusión} \label{conclulsion}

\bibliographystyle{IEEEtran}
\bibliography{references}

\end{document}
